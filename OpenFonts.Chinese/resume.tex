%%%%%%%%%%%%%%%%%%%%%%%%%%%%%%%%%%%%%%%
% Deedy - One Page Two Column Resume
% LaTeX Template
% Version 1.2 (16/9/2014)
%
% Original author:
% Debarghya Das (http://debarghyadas.com)
%
% Original repository:
% https://github.com/deedydas/Deedy-Resume
%
% IMPORTANT: THIS TEMPLATE NEEDS TO BE COMPILED WITH XeLaTeX
%
% This template uses several fonts not included with Windows/Linux by
% default. If you get compilation errors saying a font is missing, find the line
% on which the font is used and either change it to a font included with your
% operating system or comment the line out to use the default font.
% 
%%%%%%%%%%%%%%%%%%%%%%%%%%%%%%%%%%%%%%
% 
% TODO:
% 1. Integrate biber/bibtex for article citation under publications.
% 2. Figure out a smoother way for the document to flow onto the next page.
% 3. Add styling information for a "Projects/Hacks" section.
% 4. Add location/address information
% 5. Merge OpenFont and MacFonts as a single sty with options.
% 
%%%%%%%%%%%%%%%%%%%%%%%%%%%%%%%%%%%%%%
%
% CHANGELOG:
% v1.1:
% 1. Fixed several compilation bugs with \renewcommand
% 2. Got Open-source fonts (Windows/Linux support)
% 3. Added Last Updated
% 4. Move Title styling into .sty
% 5. Commented .sty file.
%
%%%%%%%%%%%%%%%%%%%%%%%%%%%%%%%%%%%%%%%
%
% Known Issues:
% 1. Overflows onto second page if any column's contents are more than the
% vertical limit
% 2. Hacky space on the first bullet point on the second column.
%
%%%%%%%%%%%%%%%%%%%%%%%%%%%%%%%%%%%%%%


\documentclass[]{deedy-resume-openfont}
\usepackage{fancyhdr}
\renewcommand{\baselinestretch}{1.4}
\pagestyle{fancy}
\fancyhf{}
    
\begin{document}

%%%%%%%%%%%%%%%%%%%%%%%%%%%%%%%%%%%%%%
%
%     LAST UPDATED DATE
%
%%%%%%%%%%%%%%%%%%%%%%%%%%%%%%%%%%%%%%
\lastupdated

%%%%%%%%%%%%%%%%%%%%%%%%%%%%%%%%%%%%%%
%
%     TITLE NAME
%
%%%%%%%%%%%%%%%%%%%%%%%%%%%%%%%%%%%%%%
\namesection{吴}{志文}{ \urlstyle{same}\href{mailto:whoami@whoareyou.com}{852905071@qq.com} | 182 1735 5389
}

%%%%%%%%%%%%%%%%%%%%%%%%%%%%%%%%%%%%%%
%
%     COLUMN ONE
%
%%%%%%%%%%%%%%%%%%%%%%%%%%%%%%%%%%%%%%

\begin{minipage}[t]{0.3\textwidth} 

%%%%%%%%%%%%%%%%%%%%%%%%%%%%%%%%%%%%%%
%     EDUCATION
%%%%%%%%%%%%%%%%%%%%%%%%%%%%%%%%%%%%%%

\section{教育经历} 
\sectionsep

\subsection{上海交通大学}
\descript{学士学位,主修软件工程}
\location{2016.09-2020.09}
\sectionsep


%%%%%%%%%%%%%%%%%%%%%%%%%%%%%%%%%%%%%%
%     LINKS
%%%%%%%%%%%%%%%%%%%%%%%%%%%%%%%%%%%%%%

\section{链接}
\sectionsep

Github:// \href{https://github.com/GIRVANAL}{\bf GIRVANAL} \\

%%%%%%%%%%%%%%%%%%%%%%%%%%%%%%%%%%%%%%
%     COURSEWORK
%%%%%%%%%%%%%%%%%%%%%%%%%%%%%%%%%%%%%%

\section{Courses}
\sectionsep
\subsection{UnderGraduate}
Operating System \\
Distributed System \\
\textbf{Introduction to Computer System} \\
\textbf{Computer System Enginering} \\
Compilers + Practicum \\
Web Developing \\
Principles of Database System  \\
\sectionsep

%%%%%%%%%%%%%%%%%%%%%%%%%%%%%%%%%%%%%%
%     SKILLS
%%%%%%%%%%%%%%%%%%%%%%%%%%%%%%%%%%%%%%

\section{技能}
\sectionsep
\subsection{编程}
\location{超过 3000 行}
C/C++ \textbullet{} Javascript \\
\location{低于 1000 行}
Python \textbullet{} Java \textbullet{} Javascript \\
MatLab \textbullet{} \\ 
\sectionsep

\subsection{前端开发}
\location{一般}
Css \textbullet{} HTML \textbullet{} \\
\location{了解}
Vue \textbullet{} React \textbullet{}\\
\sectionsep

\subsection{测试}
\location{一般}
Jmeter  \\
\sectionsep

\subsection{其他}
\location{熟悉}
git \textbullet svn \textbullet ssh \\

%%%%%%%%%%%%%%%%%%%%%%%%%%%%%%%%%%%%%%
%
%     COLUMN TWO
%
%%%%%%%%%%%%%%%%%%%%%%%%%%%%%%%%%%%%%%

\end{minipage} 
\hfill
\begin{minipage}[t]{0.68\textwidth} 

%%%%%%%%%%%%%%%%%%%%%%%%%%%%%%%%%%%%%%
%     EXPERIENCE
%%%%%%%%%%%%%%%%%%%%%%%%%%%%%%%%%%%%%%

\section{实习经历}
\sectionsep

\runsubsection{无}

\sectionsep
%%%%%%%%%%%%%%%%%%%%%%%%%%%%%%%%%%%%%%
%     RESEARCH
%%%%%%%%%%%%%%%%%%%%%%%%%%%%%%%%%%%%%%
\section{项目}
\sectionsep

\runsubsection{\bf 基于YFS的简易文件系统}
\descript{}
\location{2018.9 - 2018.12}
\vspace{\topsep}
\begin{tightemize}
    \item 实现简易文件系统的API如GET,PUT,REMOVE,CREATE 等
    \item 增加一个简易 lock server 保证多客户端对文件系统进行读写时的安全性
    \item 实现一个基于cache的lock server 
    \item 实现一个基于HDFS的简易分布式文件系统
\end{tightemize}
\sectionsep

\runsubsection{\href{https://github.com/veiasai/Xmap}{\bf Xmap}}
\descript{Main Developer}
\location{2018.7 - 2018.8}
\begin{tightemize}
    \item 主力开发, 负责后端与neo4j数据库开发
    \item 负责前端管理系统React开发
    \item 负责Jmeter对后端接口的压力测试,部分单元测试编写
    \item 点到点导航工具
    \item 结合手机罗盘、加速度仪、相机、 neo4j 图查询、微信小程序
\end{tightemize}
\sectionsep

\runsubsection{\bf 在线书店交易系统}
\descript{Owner}
\location{2018.3 - 2018.6}
\begin{tightemize}
    \item 在线书店 Web 应用
    \item 基于React的前端开发
    \item 基于MVC的设计模式,使用Spring+Struts2+Hibernete
    \item 使用 Docker容器管理Mysql,MongoDB数据
\end{tightemize}
\sectionsep

\runsubsection{\bf 简易Key-Value数据库}
\descript{Owner}
\location{2017.7}
\begin{tightemize}
    \item 使用B+树作为索引
    \item 基于C++的简易数据库
\end{tightemize}
\sectionsep
% \section{项目}
% \sectionsep
% \runsubsection{\href{https://github.com/veiasai/Xmap}{\bf Xmap}}
% \location{2018.7 - 2018.8}
% \begin{tightemize}
%     \item 负责后端与neo4j数据库开发
%     \item 前端管理系统React开发
%     \item 负责Jmeter对后端接口的压力测试,部分单元测试编写
%     \item 点到点导航工具
%     \item 结合手机罗盘、加速度仪、相机、 neo4j图查询、微信小程序
%     \end{tightemize}
% \sectionsep

% \runsubsection{\bf 在线书店交易系统}
% \descript{Owner}
% \location{2018.3-2018.6}
% \begin{tightemize}

%     \item 基于React的前端开发
%     \item Java后端开发
%     \item 基于MVC的设计模式,使用Spring+Struts2+Hibernete
%     \end{tightemize}
% \sectionsep

% \runsubsection{\bf 简易Key-Value数据库}
% \descript{Owner}
% \location{2017.7}
% \begin{tightemize}
%     \item 使用B+树作为索引
%     \item 基于C++的简易数据库
%     \end{tightemize}
% \sectionsep
%%%%%%%%%%%%%%%%%%%%%%%%%%%%%%%%%%%%%%
%     OPEN SOURCE
%%%%%%%%%%%%%%%%%%%%%%%%%%%%%%%%%%%%%%



%%%%%%%%%%%%%%%%%%%%%%%%%%%%%%%%%%%%%%
%     AWARDS
%%%%%%%%%%%%%%%%%%%%%%%%%%%%%%%%%%%%%%
\section{奖项以及证书} 
\begin{tabular}{rll}
2018         & 华东赛区三等奖  & 微信小程序开发大赛 \\
2018         & CET6 \\
\end{tabular}
\sectionsep


%%%%%%%%%%%%%%%%%%%%%%%%%%%%%%%%%%%%%%
%     PUBLICATIONS
%%%%%%%%%%%%%%%%%%%%%%%%%%%%%%%%%%%%%%

% \section{Publications} 
% \renewcommand\refname{\vskip -1.5cm} % Couldn't get this working from the .cls file
% \bibliographystyle{abbrv}
% \bibliography{publications}
% \nocite{*}

\end{minipage} 
\end{document}  \documentclass[]{article}
